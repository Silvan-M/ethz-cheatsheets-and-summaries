\section{Mehrere Zufallsvariablen}
\begin{mainbox}{}
    Die \textbf{gemeinsame Verteilungsfunktion} von $n$ Zufallsvariablen $X_1, \ldots, X_n$ (stetig oder diskret) ist die Abbildung $F: \R^n \to [0,1]$,
    $$F(x_1, \ldots, x_n) := \P(X_1 \leq x_1, \ldots, X_n \leq x_n)$$
\end{mainbox}
\subsection{Diskreter Fall - Gewichtsfunktion}
    Für $n$ diskrete ZV $X_1, \ldots, X_n$ definieren wir ihre \textbf{gemeinsame Gewichtsfunktion} $p: \R^n \to [0,1]$ durch
    $$p(x_1, \ldots, x_n) := \P(X_1 = x_1, \ldots, X_n = x_n)$$
    Dann ist die \textbf{gemeinsame Verteilungsfunktion}:
    \begin{align*}
        F(x_1, \ldots, x_n) &= \P(X_1 \leq x_1, \ldots, X_n \leq x_n)\\
        &= \sum_{y_1 \leq x_1, \ldots, y_n \leq x_n}p(y_1, \ldots,y_n)
    \end{align*}    

\begin{subbox}{Verteilung des Bildes}
Sei $n \geq 1$, $\varphi: \mathbb{R}^n \rightarrow \mathbb{R}$, $X_1, \ldots, X_n$ \textbf{diskrete} ZV mit Werten in $W_1, \ldots, W_n$. Dann ist $Z=\varphi(X_1, \ldots, X_n)$ diskret mit Werten in $W=\varphi(W_1 \times \ldots \times W_n)$. Die Verteilung von $Z$ für $z \in W$ ist:
$$
\mathbb{P}[Z=z]=\sum_{\substack{x_1 \in W_1, \ldots, x_n \in W_n \\ \varphi(x_1, \ldots, x_n)=z}} \mathbb{P}[X_1=x_1, \ldots, X_n=x_n]
$$
\end{subbox}

\textbf{Randdichte.} Seien $X_1, \ldots, X_n$ diskrete ZV mit gemeinsamer Gewichtsfkt. $p$. Für jedes $k \in\{1, \ldots, n\}$ und jedes $x \in W_k$ gilt
$$
\mathbb{P}\left[X_k=x\right]= \hspace{-0.5cm} \sum_{\substack{x_{\ell} \in W_{\ell} \\ \ell \in\{1, \ldots, n\} \backslash\{k\}}} \hspace{-0.5cm} p\left(x_1, \ldots, x_{k-1}, x, x_{k+1}, \ldots, x_n\right)
$$
\textbf{Der Erw. des Bildes der Funktion} $\varphi: \R^n \to \R$ ist
    $$\E(\varphi(X_1, \ldots, X_n)) = \sum_{x_1, \ldots, x_n}\varphi(x_1, \ldots, x_n)p(x_1, \ldots, x_n)$$
Seien $X_1, \ldots, X_n$ diskrete ZV mit gemeinsamer Verteilung $\{p(x_1, \ldots, x_n)\}_{x_1 \in W_1, \ldots, x_n \in W_n}$. Dann ist \textbf{äquivalent}:
\begin{itemize}
    \item[(i)] $X_1, \ldots, X_n$ sind unabhängig,
    \item[(ii)] für alle $x_1 \in W_1, \ldots, x_n \in W_n$ gilt
    $$
    p(x_1, \ldots, x_n)=\mathbb{P}[X_1=x_1] \cdot \ldots \cdot \mathbb{P}[X_n=x_n],
    $$
\end{itemize}
\subsection{Stetiger Fall - Gemeinsame Dichte}
\begin{mainbox}{Gemeinsame Dichte}
    Falls die gemeinsame Verteilungsfunktion von $n$ Zufallsvariablen $X_1, \ldots, X_n$ sich schreiben lässt als
    $$F(x_1, \ldots, x_n) = \int_{-\infty}^{x_1} \cdots \int_{-\infty}^{x_n}f(t_1, \ldots, t_n) \mathop{dt_n}\ldots\mathop{dt_1}$$
\end{mainbox}
\textbf{Randverteilung.} Haben $X, Y$ die gemeinsame Verteilungsfunktion $F_{X,Y}$, so ist $F_X: \R \to [0,1]$,
$$F_X(x) := \P(X \leq x) = \P(X \leq x, Y \leq \infty) = \lim_{y \to \infty}F_{X,Y}(x,y)$$
die Vertsfkt. der Randverteilung von $X$. \textit{Analog für $F_Y$.}

\textbf{Randdichte.} Sei $X,Y$ ZV mit gemeinsamer Dichte $f(x,y)$,
    $$f_X(x) = \int_{-\infty}^{\infty}f(x,y)\mathop{\text{d}y} \text{ bzw. } f_Y(y)=\int_{-\infty}^{\infty}f(x,y)\dx$$

Seien $X_1, \ldots, X_n$ ZV mit Dichten $f_{X_1}, \ldots, f_{X_n}$. \\
Dann sind folgende Aussagen \textbf{äquivalent}:
\begin{itemize}
    \item[(i)] $X_1, \ldots, X_n$ sind unabhängig,
    \item[(ii)] $X_1, \ldots, X_n$ sind gemeinsam stetig mit gemeinsamer Dichte $f: \mathbb{R}^n \rightarrow \mathbb{R}_{+}$,
    \item[]\hspace{1cm}d.h. die gemeinsame Dichtefunktion $f$ ist das Produkt der einzelnen Randdichten $f_{X_k}$, also
    $$
    f\left(x_1, \ldots, x_n\right)=f_{X_1}\left(x_1\right) \cdot \ldots \cdot f_{X_n}\left(x_n\right).
    $$
\end{itemize}
